

\documentclass{article}
\usepackage[utf8]{inputenc}
\usepackage{setspace}
\usepackage{ mathrsfs }
\usepackage{amssymb} %maths
\usepackage{amsmath} %maths
\usepackage[margin=0.2in]{geometry}
\usepackage{graphicx}
\usepackage{ulem}
\setlength{\parindent}{0pt}
\setlength{\parskip}{10pt}
\usepackage{hyperref}
\usepackage[autostyle]{csquotes}

\usepackage{cancel}
\renewcommand{\i}{\textit}
\renewcommand{\b}{\textbf}
\newcommand{\q}{\enquote}
\newcommand{\p}{$\phi \ $}

\renewcommand{\H}{\Bbb H}
%\renewcommand{\l}[1]{\lceil #1 \rceil }
\renewcommand{\l}[1]{( #1 ) }
%\newcommand{\lte}{\sqsubset}
%\newcommand{\lte}{\angle}
\newcommand{\lte}{\preceq}




%\vskip1.0in



\begin{document}
\begin{huge}
{\setstretch{0.0}{

LADDERS %Ħ

\section{Definition}
Let $\Bbb H$ be the set of ladders.  A ladder is a set of positive rationals $\{f(n)\}_{n=1}^{\infty}$ where $f : \Bbb N \to \Bbb Q $ and

$$\exists q_0 \in \Bbb Q\ \ \forall n \in \Bbb N \quad  0 < f(n) < f(n+1) < q_0$$. 



The function $f$ is a \q{constructor} or \q{climber} of the ladder, and we can write the ladder more simply as $\l{f}$. We can even just refer to the ladder $f$ when the context makes it clear that what we intend is the set generated by $f$. The set is introduced to encourage a visualization of $\l{f}$ as a ladder with infinitely many rungs -- something we might use to measure heights with irrational magnitudes. 

Note that a ladder's rungs get arbitrarily dense as one constructs or climbs it, though of course ladders, because their sequences are strictly increasing, never have a top or final rung. A rational condensation point may exist, but it is never of rung on the same ladder.  

There's something nightmarish about absurd about these ladders, for one can climb them forever without ever getting beyond that $q_0$ (or of course even to the rational limit, when it exists.)
\section{Order} 


We define $\l{f}  \le \l{g} $ when $\forall q \  \exists p \quad f(q) < g(p)$

We define $\l{f} \sim \l{g}$ (or $\l{f} = \l{g}$) when $\l{f} \le \l{g} $ and  $\l{g} \le \l{f}$. This is the \q{tortoise and hare} condition. If either sequence stays put, the other will eventually pass beyond it, only to be passed itself, if \i{it} halts. 
 

We define $\l{f} < \l{g} $ when $\exists m \ \forall p \quad f(p) < g(m)$.

We always have, for any $\l{f},\l{g} \in \H$, either $\l{f}  < \l{g} $, $\l{g}  < \l{g} $, or $\l{f}  = \l{g} $, for the negation of $<$ and $>$ is logically equivalent to $\sim.$
\section{Addition and Multiplication}

We define $\l{f} + \l{g} = \l{f + g } = \{f(n) + g(n) \}_{n=1}^{\infty}.$ 

We define $\l{f} + \l{g} = \l{fg} = \{f(n)g(n) \}_{n=1}^{\infty}.$ 

Both this sum and product are easily seen to be an elements of $\H$. This comes easy with the use of strictly increasing sequences, but it makes defining subtraction more complicated. 

Note that one can follow the construction of integers in set theory to create additive inverses for ladderlike entities, but one is then dealing (at the least) with \i{pairs} of ladders, with the second ladder intuitively representing the negativity of the pair. If the goal is achieving a construction of the real number system, this is perhaps (?) the best we can do. But the system $\H$ in intended as a \q{realish} number system -- one that has square roots and even more exotic limits. So we get the basic \q{magic} of the real number system with the extra trouble that comes with negative numbers and subtraction. 

\section{Least Integer Tricks}

If $\l{f} < \l{g}$ then there is a \b{least} $k$ such that $\forall p \ f(p) < g(k)$. This the first value of $g$ that $f$ can never equal or surpass. Define $m$ to be the least such $k$. Then $\forall p \ f(p) < g(m) < g(m + 1) < g(m + 2) < ...$. 

This can be useful in proofs. It's also nice to keep definitions as concrete and computable as possible.

A similar trick takes the some typical set theory ordering of the positive rational numbers and then chooses the least $k$ such that $q_k \in \Bbb Q$ satisfies some desirable property. }}

\section{Limits}

Now we consider an a strictly increasing but bounded sequence of ladders. So we have $\l{f_1}  < \l{f_2}  < \l{f_3} < ... < \l{g}$.  

An example: Let $g$ might be something like $g(p) = 10 - \frac{1}{p}$, and let $f_n(p) = 10 - \frac{1}{p} - \frac{1}{n}$. Then we have $g(p) \to 10$ and $f_n(p) \to 10 - \frac{1}{n}$ as $p \to \infty$.

Intuitively, we have $\l{f_n} \to \l{g}$, but we haven't defined the distance between ladders, and of course we don't have subtraction. We just have order.


Using just this order, we can define something that is half limit and half least upper bound for such a sequence. It might be called a \q{crown} or a \q{cap}. I'll use $\l{\bar f}$.

Given an increasing, bounded sequence of ladders, we can get a specific $\l{\bar f}$ by using a least integer trick mentioned above. 

For each $n$,  $m_n$ as the least $m$ such that $\forall p \ \ f_{n-1}(p) < f_n(m)$. 

Now we can define $\bar f(n) = f_n(m_n)$. 

Note that $\l{\bar f}$ is an upper bound, since 

$\forall n \quad \forall p \quad f_n(p) < f_{n+1}(m_{n+1}) = \bar f(n+1)$.

So $\bar f$ eventually surpasses $f_n(p)$ for any fixed $n$ and fixed $p$.

We can also show that $\forall n \ \l{f_n} < \l{h} \implies \l{\bar f} \le \l{h}$.

For any $n$, there exists $k_n$ such that $\forall p \ \ f_n(p) < h(k_n)$. 

So $ \bar f(n) = f_n(m_n) < h(k_n)$, so $h$ can always surpass $\bar f$ for any fixed $n$.  So 

We have shown then that $\l{f_n}  < \l{f_{n+1}}  <  ... < \l{\bar f} \le \l{h}$, for \i{any} upper bound $\l{h}$.   


It therefore sensible to \b{define} $\lim_{n \to \infty} [ \l{f_n} ]  =  \l{\bar f}$. Or we can just use the notation and concept of $\l{\bar f}$ without reference to real analysis. 

\section{Uses of Limits : Classic Irrational Numbers}
One use of this concept might be establishing something like $\sqrt{2}$ in $\H$. This could be done more directly, by using a sequence of rational numbers that converges to $\sqrt{2}$ in the real numbers, but we could also work with a homomorphism from the positive rational numbers into our system of ladders. 

For instance, we can map $q \in \Bbb Q$ with $\phi$ so that $\phi(q) = \{\frac{n}{n+1}q\}_{n=1}^{\infty}$. We can then use an increasing sequence $q_k$ such that $1 < q_{k-1}^2 < q_k^2 < 2 < q_k^2 + 2^{-k}$. To use our notation from the previous section, we let $f_k(n) = \frac{n}{n+1}q_k$. 

Then, for any fixed $p$, there exists an $n$ such that $\frac{p}{p+1}2 < \bar f ^ 2 (n) < 2 < \bar f ^ 2 (n) + 2^{-n}$. 

So $\phi(2) = \{\frac{n}{n+1}2\}_{n=1}^{\infty} \le \l{\bar f}^2 = \l{\bar f ^2}.$ We can easily reverse the role of $p$ and $n$ to get $\l{\bar f}^2 \le \phi(2)$, so  $\l{\bar f}^2 \sim \phi(2)$.

\section{Transcendental Numbers}

It's not difficult to create the analogue of $e$ in our system, since $e$ is the limit of an increasing sequence of positive rational numbers. We don't need subtraction. We define $f(n)$ as the sum of the first $n$ terms of the sequence $1,1,\frac{1}{2}, \frac{1}{6}, ...$. Then we take $\phi(f(n))$ as our sequence of ladders converging to the ladder version of $e$.

\section{Notes}
I left out many easy proofs that I may come back and add. I struggled to define something like subtraction or a multiplicative inverse, but so far no luck. Feel free to email me at my-github-name@proton.me, if you want to talk math.

\end{huge}
\end{document}
